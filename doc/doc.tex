% This file is part of matrix2latex.

% matrix2latex is free software: you can redistribute it and/or modify
% it under the terms of the GNU General Public License as published by
% the Free Software Foundation, either version 3 of the License, or
% (at your option) any later version.

% matrix2latex is distributed in the hope that it will be useful,
% but WITHOUT ANY WARRANTY; without even the implied warranty of
% MERCHANTABILITY or FITNESS FOR A PARTICULAR PURPOSE.  See the
% GNU General Public License for more details.

% You should have received a copy of the GNU General Public License
% along with matrix2latex. If not, see <http://www.gnu.org/licenses/>.

\documentclass[12pt, a4paper, reqno]{report}
%\documentclass[a4paper, english]{IEEEtran}
\usepackage[english]{babel}
\usepackage[utf8]{inputenc}

%\usepackage[cmex10]{amsmath}    % Recommended by IEEEtran
%\interdisplaylinepenalty=2500   % Recommended by IEEEtran, long math
\usepackage{amsmath}
\usepackage{amsfonts, amssymb, amsthm}
\allowdisplaybreaks[1] % displaybreak in math is allowed but avoided

\usepackage{graphicx}           % include figures
\usepackage{hyperref}           % links
\usepackage[amssymb]{SIunits}   % provides SI suffixes
\usepackage{units}              % provides \unit[1]{\centi\meter}
\usepackage{booktabs}           % Fancy tables

\usepackage{pythontex}
\setpythontexworkingdir{.}

\usepackage{listings}           % include code
\lstset{language = python,
  showspaces = false,
  showtabs = false,
  showstringspaces = false,
  mathescape = true}


% Useful macros
\providecommand{\fixme}[1]{\textbf{{\textsc{FIXME:} #1 }}}

% Math
\providecommand{\e}[1]{\ensuremath{\times 10^{#1}}} % scientific notation

\author{Øystein Bjørndal}

\title{Documentation for matrix2latex}

\begin{document}
\maketitle
Takes a python matrix or nested list and converts to a LaTeX table or matrix.
Author: ob@cakebox.net

This software is published under the GNU GPL, by the free software
foundation. For further reading see: 
\url{http://www.gnu.org/licenses/licenses.html#GPL}

The following packages and definitions are recommended in the latex preamble 
\begin{verbatim}
% scientific notation, 1\e{9} will print as 1x10^9
\providecommand{\e}[1]{\ensuremath{\times 10^{#1}}}
\usepackage{amsmath} % needed for pmatrix
\usepackage{booktabs} % Fancy tables
...
\begin{document}
...
\end{verbatim}

\section{Arguments}
  
\subsection{matrix}
  A numpy matrix or a nested list
  TODO:
  - Any python structure that looks like a rektangular matrix.
  - Remove dependency on numpy (might be more portable to other systems)

\subsection{Filename}
  File to place output, extension .tex is added automatically. File can be included in a LaTeX
  document by \verb!\input{filename}!. If filename is None or not a string, output will be returned in a string
  
\subsection{*environments}
  Use 
\lstinline{matrix2latex(m, None, "align*", "pmatrix", ...)} for matrix.
  This will give
  \begin{align*}
    \begin{pmatrix}
    \end{pmatrix}
  \end{align*}
  Use 
\lstinline{matrix2latex(m, "test", "table", "center", "tabular" ...)} for table.
  Table is default so given no arguments: table, center and tabular will be used.
  The above command is then equivalent to
  matrix2latex(m, "test", ...)
  
\subsection{**keywords}
\subsubsection{transpose}
  Flips the table around in case you messed up. Equivalent to
  matrix2latex(m.H, ...)
  if m is a numpy matrix.
  
\subsubsection{format}
  Printf syntax format, e.g. \lstinline{$%.2f$}. Default is \lstinline{$%g$}.
  This format is then used for all the elements in the table.
  
\subsubsection{formatColumn}
  A list of printf-syntax formats, e.g. \lstinline{[$%.2f$, $%g$]}
  Must be of same length as the number of columns.
  Format i is then used for column i.
  
\subsubsection{alignment}
  Used as an option when tabular is given as enviroment.
  \verb!\begin{tabular}{alignment}!
    A latex alignment like c, l or r.
    Can be given either as one per column e.g. "ccc".
    Or if only a single character is given e.g. "c",
    it will produce the correct amount depending on the number of columns.
    Default is "r".

\subsubsection{rowLabels}
    A row at the top used to label the columns.
    Must be a list of strings.

\subsubsection{columnLabels}
    A column used to label the rows.
    Must be a list of strings

\subsubsection{caption}
    Use to define a caption for your table.
    Inserts \verb!\caption! after \verb!\end{tabular}!.

\subsubsection{label}
  Used to insert \verb!\label{...}! after \verb!\end{tabular}!
Default is filename without extension.

Both caption and label will do nothing if tabular environment is not used.
\begin{sagesilent}
import sys
sys.path.append('../')
\end{sagesilent}
\begin{sageblock}
from matrix2latex import matrix2latex
from numpy import matrix
m = matrix("1 2 4;3 4 6") # numpy matrix or
m = [[1, 2, 4], [3, 4, 6]] # python nested list
matrix2latex(m, "test", "table", "center", "tabular", format="$%.2f$", alignment="lcr")
# or since table, center and tabular is default:
t = matrix2latex(m, format="$%.2f$", alignment="lcr")
# produces:
\end{sageblock}
\sage{t}

\section{Usage examples}

\subsection{Minimal}
\begin{sageblock}
from matrix2latex import matrix2latex
m = [[1, 2, 3], [1, 4, 9]] # python nested list
t = matrix2latex(m)
print t
\end{sageblock}
\sage{t}

\subsection{Labels}
Using the same minimal example from above we can add row labels
\begin{sageblock}
rl = ['$x$', '$x^2$']
t = matrix2latex(m, rowLabels=rl)
\end{sageblock}
\sage{t}

Or swapping it around
\begin{sageblock}
cl = ['$x$', '$x^2$']
t = matrix2latex(m, columnLabels=cl, transpose=True)
\end{sageblock}
\sage{t}

\subsection{Caption}
We can easily add a caption
\begin{sageblock}
t = matrix2latex(m, rowLabels=rl, 
                 caption='Nice table!')
\end{sageblock}
\sage{t}

\subsection{Label}
We can use \lstinline{label='niceTable'} but if we save it to file
the default label is the filename, so:
\begin{sageblock}
matrix2latex(m, 'niceTable', rowLabels=rl, 
                 caption='Nice table!')
\end{sageblock}
can be referenced by \verb!\ref{niceTable}!. Table \ref{niceTable}
was included in latex by \verb!\begin{table}[ht]
	\begin{center}
	\caption{Nice table!}
	\label{tab:niceTable}
		\begin{tabular}{cc}
		\toprule
			$x$ & $x^2$\\
			\midrule
			$1$ & $1$\\
			$2$ & $4$\\
			$3$ & $9$\\
		\bottomrule
		\end{tabular}
	\end{center}
\end{table}!.
\begin{table}[ht]
	\begin{center}
	\caption{Nice table!}
	\label{tab:niceTable}
		\begin{tabular}{cc}
		\toprule
			$x$ & $x^2$\\
			\midrule
			$1$ & $1$\\
			$2$ & $4$\\
			$3$ & $9$\\
		\bottomrule
		\end{tabular}
	\end{center}
\end{table}

TODO: add simple real world example.
% \begin{table}[ht]
%   \begin{center}
%     \begin{tabular}{lcr}
%       $1.00$ & $2.00$ & $4.00$\\
%       $3.00$ & $4.00$ & $6.00$\\
%     \end{tabular}
%   \end{center}
% \end{table}


%\bibliographystyle{IEEEtran}
%\bibliography{doc}
\end{document}
